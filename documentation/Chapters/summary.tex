\chapter*{Summary}
\addcontentsline{toc}{chapter}{Summary}
Nowadays traffic congestion and traffic jams got a great attention. Understanding the causes of these issues may help to solve them. The causes could be the inattention of the drivers, hurrying and other activities. One of the possible solution could be autonomous vehicles. These could exclude the human behavior from the traffic and possibly improve the situation. The goal of this document was to investigate and create models for human drivers and autonomous vehicles. Implement a simulator based on the mathematical models. Simulate different cases and investigate the behavior of the models.

A literature review has been accomplished to get a better overview about the traffic models and the possibilities of simulating traffic situations. Based on literature traffic models can be sorted into two groups. There are macroscopic and microscopic models. Macroscopic traffic flow models describe the traffic as a fluid. This has some analogies in the physics of fluids. In contrast to the macroscopic models microscopic models consider every car in the traffic flow. An example of the latter is the Intelligent Driver Model.

Intelligent Driver Model defines the acceleration value for each car at each time step based on all neighboring vehicles. It has two states, the follower and the free acceleration state. In follower state the vehicle maintains the safety headway from its leader car and its speed equals to the leader's speed. However in free acceleration state the vehicle can accelerate to its desired speed.

Lane changes have been modeled with the so called MOBIL algorithm. The fundamental idea behind this is that each driver considers the local traffic (all neighboring vehicles) and changes lanes if the local situation improves. Naturally there are drivers considering this more then others. There are customizable parameters for each driver to take into account this behavior in the model.

The simulator has been implemented using the previously mentioned algorithms. The simulator has been test with cars at a red traffic light turning into green. It turned out that the simulator has neglected some important part of the traffic flow. One of the most important one is that a lane change could last for multiple seconds and does not happen instantly. Another one is that the model cannot simulate human behavior like inattention. To solve these issues a suggested algorithm has been implemented. The experience was that the vehicles reached slower the same distance then before. Finally an other case was implemented where one of the lanes was occupied for an area forcing drivers to change lanes.

Self-driving cars can be simulated in the simulator with neglecting the human behaviors. Based on this the cases were simulated again. However some of the cars were exchanged to autonomous vehicles. At the end all of the vehicles were self-driving. For the first case where there was no obstacle the result was that one self-driving car can improve the traffic flow if it is positioned correctly. However in the second case the effect of the autonomous vehicle depended on its position. If positioned poorly it could dis improve the traffic flow as well.