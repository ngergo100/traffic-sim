\chapter*{Összefoglalás}
\addcontentsline{toc}{chapter}{Összefoglalás}
Napjainkban nagy figyelmet kapnak a közlekedési dugók, annak okainak felismerése, illetve azok mentén a probléma feloldása. Az okok többek között lehetnek az autósok figyelmetlensége, a sietés, illetve más egyéb tevékenységek. Az egyik ilyen lehetséges feloldási mód akár lehetne az önvezető autók bevezetése amivel az emberi tényezőt ki lehetne hagyni és ezzel valószínűsíthetően javítani a helyzeten. A tézis célja többek között a sofőrök illetve az autonóm járművek modelljének felállítása. A matematikai modellek alapján egy szimulátor implementálása, majd különböző szituációk szimulálása által ezek megismerése.

Egy átfogó kép kialakításáért irodalom kutatást végeztem a közlekedési modellekről és közlekedési szituációk szimulációs lehetőségeiről. Ezek alapján két csoportba oszthatók a közlekedési modellek. Léteznek makroszkópikus illetve mikroszkódikus modellek. A makroszkópikus modellek egyfajta áramlástani folyamatként kezelik a forgalmat analógiát vonva ezzel a folyadékok és gázok dinamikájával. A mikroszkópikus modellek azonban egyénileg kezelnek minden egyes járművet a közlekedésben. Utóbbira egy példa az Intelligent Driver Model ami az elkészült szimulátor alapjául szolgált.

Az Intelligent Driver Model az egyes járművek pillanatnyi gyorsulását adja meg a lokális környezetének függvényében. Két állapota van a modellnek attól függően, hogy van-e egy másik jármű előtte vagy sem. Ez a két állapot a követő mód és a szabad gyorsulási mód. A követő módban az adott jármű fenntartja a biztonságos követési távolságot és az előtte haladó autó sebességét. Azonban szabad gyorsulási módban a sofőr az általa megválasztott sebességet fogja tartani.

A sáv váltásokat az ún. MOBIL algoritmus modellezte. Ennek lényege, hogy minden egyes sofőr figyelembe veszi a körülötte lévő közlekedési szituációt és úgy dönt hogy lokális helyet javuljon. Természetesen vannak sofőrök akik ezt jobban szem előtt tartják míg mások kevésbé. Ennek a figyelembe vételére szolgálnak a modell paraméterei amiket az egyes sofőrökhöz lehet rendelni.

A két említett algoritmus felhasználásával a szimulátor implementálásra került. Egy kétsávos úton piros lámpánál álló majd elinduló közlekedési szituáción keresztül lett tesztelve elsőként. Kiderült, hogy a szimulátor ezen formájában még sok pontatlanság található. Az egyik legfontosabb ilyen, hogy a sáv váltások több másodpercig eltartó folyamatok és nem pillanatszerűek.  Egy másik hasonlóan fontos hiba, hogy jelen formájában nem képes szimulálni az emberi tényezőket mint például a figyelmetlenség. A problémákra egy javasolt algoritmus bekerült a szimulátorba majd ugyanazon a szituáción tesztelve lett. A tapasztalat az, hogy az autók így lassabban érték el ugyanazt a távot mint ez ezelőtt. Végezetül egy másik eset is fel lett állítva ahol a ugyanaz volt a kiindulási eset azonban az egyik sáv egy szakaszon nem volt járható így sávváltásra kényszerítve az autókat.

Az önvezető autókat az elkészült szimulátorban úgy lehet modellezni, hogy az emberi tulajdonságait az adott sofőrnek kikapcsoljuk. Ezek alapján újra le lettek futtatva az előző esetek úgy, hogy fokozatosan egyre több és több önvezető tulajdonságú autó vett részt az esetben az emberi tulajdonságúak helyett. Az első esetben ahol nem volt akadály a tapasztalat az, hogy akár már egy önvezető tulajdonságú autó is sokat jelenthet a gyorsabb haladás szempontjából. Míg a második esetnél az autonóm járművek attól függően, hogy milyen pozícióba kerülnek akár csökkenthetik is a közlekedés sebességét.