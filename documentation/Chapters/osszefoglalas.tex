\chapter*{Összefoglalás}
\addcontentsline{toc}{chapter}{Összefoglalás}
Napjainkban nagy figyelemet kapnak a közlekedési dugók és annak okainak felismerése, illetve azok mentén a probléma felodása. Az okok többek között lehetnek az autósok figyelmetlensége, a sietés, illetve más egyéb tevékenységek. Az egyik ilyen lehetséges feloldási mód akár lehetne az önvezető autók bevezetése amivel az emberi tényezőt ki lehetne hagyni és ezzel valószínűsíthetően javítani a helyeten. A tézis célja többek között a sofőrök illetve az autonóm járművek modelljének fellálítása majd különböző szituációkon szimulálása által ezek megismerése.

Egy átfogó kép kialakításáért irodalomkutatást végeztem a közlekedési modellekről és közlekedési szituációk szimulációs lehetőségeiről. Ezek alapján két csoportba oszthatók a közlekedési modellek. Léteznek makroszkópikus illetve mikroszkódikus modellek. A makroszkópikus modellek egyfajta áramlástani folyamatként kezelik a forgalmat analógiát vonva ezzel a folyadékok és gázok dinamikájával. A mikroszkópikus modellek azonban egyenélig kezelnek minden egyes járművet a közlekedésben. A két modellcsoport közötti összefüggés  nem meghatározott egyértelműen. Az utobbira egy példa az Intelligent Driver Model.

Az Intelligent Driver Model az egyes járművek pillanatnyi gyorsulását adja meg a lokális környezetének függvényében. Attól függően, hogy vannak e egy jármű előtt illetve milyen távolságra mekkora sebességel halad a jármű lehet követő módban is a modell illetve szabad gyorsulási módban. A követő módban az adott jármű fenntartja a biztonsagos követési távolságot ás