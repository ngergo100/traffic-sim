\chapter{Introduction}
Recently there is huge attention on self-driving cars. It seems that in a few years self-driving cars will hit the public roads. So it could be interesting that how much effect these cars will have on today's crowded streets. To find out a model should be developed to simulate real traffic situations without autonomous cars and then exchange real drivers one by one to `robots`. To develop a simulator that is capable of simulating real traffic behaviors it was a must to check out what has been already developed by the industry.

There are several publications and even open source example implementations on this topic. Various traffic flow models have been developed like car following or hydrodynamic models. One of the main applications is that based on the traffic simulation results provided by these simulators, engineers can design better traffic systems. It can be also used to simulate stop and go traffic on the highways and suggest a speed limit based on that to avoid traffic jam. There are researches about improving traffic via network communications between cars and also with traffic lights. These researches also based on traffic simulators.

Nowadays every car has some kind of navigation system which considers the traffic congestion. Most of them is community driven mobile phone applications like Waze or Google Maps. However they are not able consider the local traffic situations which can be important too. There are a lot of traffic data which could be used to advise actions to individual drivers to make the local traffic more fluent. The question is that what should be advised. So the aim of the document is to create a simulator which is capable of simulating a real traffic situations so at least the problem can be investigated.

In order to create this simulator we used models available in the literature but we also made some improvements on the models. Namely, there are several not considered properties of human driving that can have relevant effects on the traffic flow. The difference in the traffic flow developed in case of human drivers and in case of autonomous cars strongly depend on the human factors. For example the divided attention of the human control often lead to bad traffic conditions or in some cases to accidents. In this research, in the simulation code our human drivers can focus separately on longitudinal, lateral dynamics and other activities.

In this thesis  two traffic situation will be presented to show that without any specific strategy how self-driving cars can improve the traffic flow.