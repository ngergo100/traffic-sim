\begin{abstract}
Nowadays traffic in rush hours is crowded. The causes of this issue - besides the countless automobiles which is much more than the streets of cities were initially designed for - are inattention, hurrying and other distracting activities. This report purpose is to investigate the effect of autonomous drivers in the rush hour traffic with special attention to the red light situations.

A literature review was accomplished to investigate the behavior of normal and autonomous drivers. Based on that it can be stated that traffic models can be divided into micro- and macroscopic models. Microscopic behaviors are more suitable to model red light situations, so the process was continued with these.

The red light situations were simulated with a car following model. The algorithm used during the simulations was a combination of Intelligent Driver Model and the so called MOBIL model.

Several simulations with various driver types (form the low reaction time old timer to the aggressive business man) were evaluated.

The simulations showed that the effect of autonomous drivers is very positive in traffic.
\end{abstract}