\chapter{Behavior of self driving vehicles}
		Nowadays almost every new car has some level of automated driving assistant feature.
	\section{Model of a self driving car}
		So the task is to find a model of an autonomous driver. The key observation is that an self driven car can be modeled with the same model as used before. The only difference relies on IDM and MOBIL parameters. So the task is to find proper parameters for the model that can be used for autonomous driver simulation.

		There are three main components representing human behavior built in the model currently. The first one is the attention for driving. An autonomous driver always paying attention to the road. The other component is that a self driving car can always pay attention to both longitudinal and transversal motions. They are not limited to longitudinal motion attention when they have a higher acceleration or deceleration value. The last but not least component that it can have more favorable IDM and MOBIL parameters than most of the non self driving cars. E.g. an autonomous driver will always accelerate with the greatest possible value which is comfortable, they would not change lanes if the advantage they would get be marginal however would put others significantly worst position.